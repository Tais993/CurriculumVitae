%%%%%%%%%%%%%%%%%%%%%%%%%%%%%%%%%%%%%%%%%
% Twenty Seconds Resume/CV
% LaTeX Template
% Version 1.0 (14/7/16)
%
% Original author:
% Carmine Spagnuolo (cspagnuolo@unisa.it) with major modifications by
% Vel (vel@LaTeXTemplates.com), Harsh (harsh.gadgil@gmail.com),
% ZabuzaW (zabuza.dev@gmail.com) and Tais993
%
%%%%%%%%%%%%%%%%%%%%%%%%%%%%%%%%%%%%%%%%%

%----------------------------------------------------------------------------------------
%	PACKAGES AND OTHER DOCUMENT CONFIGURATIONS
%----------------------------------------------------------------------------------------

\documentclass[letterpaper]{twentysecondcv} % a4paper for A4

%----------------------------------------------------------------------------------------
%	 PERSONAL INFORMATION
%----------------------------------------------------------------------------------------
% If you don't need one or more of the below, just remove the content leaving the command, e.g. \cvnumberphone{}

\cvname{Tijs Beek} % Your name
\cvjobtitle{Software Developer, 16} % Job
\cvavatar{avatar.jpg} % Avatar
% title/career

\cvaddress{Uden, Nederland} % Address
\cvlinkedin{/in/tijs-beek-ab038217b/} % LinkedIn site
\cvgithub{Tais993} % Github site
\cvgithubRepos{28} % GitHub amount of repositories
\cvgithubForks{2} % GitHub amount of forks
\cvgithubStars{3} % GitHub amount of stars
\cvsite{tais993.github.io/} % Personal website
\cvnumberphone{} % Phone number
\cvstackoverflow{} % StackOverflow user name
\cvstackoverflowId{} % StackOverflow user id
\cvstackoverflowRep{} % StackOverflow user reputation
\cvstackoverflowGold{} % StackOverflow amount of gold badges
\cvstackoverflowSilver{} % StackOverflow amount of silver badges
\cvstackoverflowBronze{} % StackOverflow amount of bronze badges
\cvmail{} % Email address


% Programming skill bars
\programming{
	{HTML $\textbullet$ CSS $\textbullet$ JavaScript $\textbullet$ PHP / 2},
	{C\# / 4},
	{Java / 6}}

% Language overview bubbles
\newcommand\languages{
~
	\smartdiagram[bubble diagram]{
		\textbf{~~~Nederlands~~~}\\\textbf{oorspronkelijk},
		\textbf{~~~~~~~~Engels~~~~~~~~}\\\textbf{B2}
	}
}

% Technologies
\technologies{
	{materialteal!35/Java/Gson, Jackson, Jetbrains Annotations\vspace{1mm}\\
		JDA, Caffeine, JUnit, Logback, SLF4J, \vspace{1mm}\\
		Log4j},
	{materialteal!35/C\#/WPF, XUnit, Moq, NewtonsoftJson, \vspace{1mm}\\
	Jetbrains Annotations},
	{materialteal!35/Spring/Boot, Framework, Web, WebClient, \vspace{1mm}\\
	WebFlux, Reactor},
	{materialteal!35/Frontend/TailwindCSS, ReactJS},
	{materialteal!35/Databases/MongoDB},
	{materialteal!35/Networking/REST, JSON},
	{materialamber!20/IDEs/IntelliJ, Rider, PhpStorm, VSC, \vspace{1mm}\\
	Visual Studio},
	{materialamber!20/Versioning/Git, GitHub, GitLab, GitKraken},
	{materialamber!20/Build Management/Gradle, Maven, Resharper},
	{materialamber!20/Organization/Github, Space},
	{materialamber!20/Communication/Slack, Discord, Skype, MS Teams, MS Outlook},
	{materialamber!20/Image $\textbullet$ Video/Gimp, paint.net, OBS Studio}}

%----------------------------------------------------------------------------------------

\begin{document}

\makesidebarFirst % Print the sidebar

%----------------------------------------------------------------------------------------
%	 Profile
%----------------------------------------------------------------------------------------
\section{Profiel}

\begin{itemize}
	\item \textbf{Twee jaar} programmeer ervaring
	\item \textbf{Expert in Java}, kennis van C\# en gemiddelde Fullstack kennis
	\item \textbf{Extreem} gepassioneerd in programmeren
\end{itemize}

\vspace{6mm}


%----------------------------------------------------------------------------------------
%	 EDUCATION
%----------------------------------------------------------------------------------------
\section{Onderwijs}

\begin{twenty} % Environment for a list with descriptions
	\twentyitem
    	{2021 - tegenwoordig~~~~}
        {}
        {Software developer (bol)}
        {\href{https://www.summacollege.nl/}{Summa College, Eindhoven}}
        {}
        {}
	%\twentyitem{<dates>}{<title>}{<organization>}{<location>}{<description>}
\end{twenty}


%----------------------------------------------------------------------------------------
%	 Publications
%----------------------------------------------------------------------------------------

\section{Publicaties}
\begin{twenty} % Environment for a list with descriptions
	%\twentyitem{<dates>}{<title>}{<location>}{<description>}
\end{twenty}

%----------------------------------------------------------------------------------------
%	 Hobbies
%----------------------------------------------------------------------------------------
\section{Hobbies}

\begin{twenty} % Environment for a list with descriptions
	\twentyitem
    	{2012 -}
        {tegenwoordig~~~~}
        {Scouting Uotha}
        {\href{https://www.uotha.nl/}{Scouting Uotha}}
        {}
            { Ben er jaren geleden mee begonnen, en ik ga er nog steeds naartoe meteen grote glimlach.}\\
	\twentyitem
    	{Feb 2019 -}
        {tegenwoordig~~~~}
        {Programmeren}
        {\href{https://tais993.github.io/}{GitHub Pages}}
        {}
        {Bijna 2 jaar geleden begonnen met actief programmeren, pas in Augustus van 2021 begonnen met Software Development op school.}
\end{twenty}

\vspace{6mm}


%----------------------------------------------------------------------------------------
%	 Experience
%----------------------------------------------------------------------------------------

\section{Ervaringen}
\begin{twenty} % Environment for a list with descriptions
	\twentyitem
    		{Feb 2021 -}
		{tegenwoordig}
        		{Bezorger bij de Kwalitaria}
        		{\href{https://kwalitaria.nl/uden-zuid/uden-zuid/}{Kwalitaria, Uden-Zuid}}
        		{}
        		{Voedselleveringen}\\
		\twentyitem
    		{Mar 2018 -}
		{Apr 2021}
        		{Leerlingenraad Udenscollege}
        		{\href{https://www.udenscollege.nl/vmbo/startpagina-vmbo/}{Udenscollege VMBO}}
        		{}
        		{
        		    Voor 3 jaar ben ik lid geweest van de leerlingen raad op onze school.
        		    Langer dan 1 jaar in deze tijd ben ik voorzitter geweest, ik stopte hiermee toen ik in mijn examen jaar ging.
        		}
	%\twentyitem{<dates>}{<title>}{<location>}{<description>}
\end{twenty}

\vspace{6mm}


%----------------------------------------------------------------------------------------
%	 Projects
%----------------------------------------------------------------------------------------

\section{Projecten (1 of 5)}
\begin{twenty} % Environment for a list with descriptions
	\twentyitem
    		{Oktober 2021 -}
		{Present}
        		{Java-OsuApiWrapper}
        		{\href{https://github.com/Tais993/Java-OsuApiV1/}{GitHub/Java-OsuApiV1}}
        		{Java 17, Gradle, Spring Boot Web, WebClient, Reactor Spring, Caffeine, IntelliJ Annotations, Jackson}
				{
				Een wrapper voor de osu!api, dit is een REST API voor osu! de game
				Rework van een van mijn oudere projecten.

				Dit project probeert te focusen op code kwaliteit door gebruik te maken van o.a. het SOLID principle.
				\begin{itemize}
					\item Spring, WebClient
					\item Reactor, Reactive Streams
					\item Zeer aanpasbaar
					\item Duidelijke documentatie en tutorials
				\end{itemize}}\\

\end{twenty}

\newpage

\makesidebarSecond % Print the sidebar

%----------------------------------------------------------------------------------------
%	 Projects
%----------------------------------------------------------------------------------------

\section{Projecten (4 of 5)}
\begin{twenty} % Environment for a list with descriptions
	\twentyitem
    		{Aug 2021 -}
		{tegenwoordig}
        		{TJ-Bot}
        		{\href{https://github.com/Together-Java/TJ-Bot/}{GitHub/TJ-Bot}}
        		{Java 17, Gradle, Jetbrains Annotations, JDA, JUnit, SLF4J, Logback, SQLite, Sonarcloud, Spotless}
        		{Discord bot voor de zogenaamde "Together Java" Discord server, de bot wordt onderhouden door de community.}\\
	\twentyitem
    		{Sep 2021 -}
		{Oct 2021}
        		{VeelPlezier}
        		{\href{https://github.com/Tais993/VeelPlezier}{GitHub/VeelPlezier}}
        		{C\#, .NET framework 7.2, WPF, JetBrains Annotations, NewtonsoftJson, XUnit, Moq}
        		{
        		GUI kassa
        		\begin{itemize}
        			\item Responsieve GUI
        			\item Lokalisatie
        			\item Inclusief rekenmachine
        			\item Inclusief valuta omzetter
        		\end{itemize}}\\
	\twentyitem
    		{Sep 2020 -}
		{Dec 2020}
        		{TaisDiscordBot}
        		{\href{https://github.com/Tais993/taisdiscordbot}{GitHub/Cobweb}}
        		{Java 15, JDA, Lavaplayer, MongoDB, Logback, Google API services YouTube, Google API Client, Json-simple}
        		{Basis Discord bot met 50+- commands}\\
	\twentyitem
    		{June 2020 -}
		{Sep 2021}
        		{TorchCraftExcelMod}
        		{\href{https://github.com/Tais993/TorchCraftExcelMod}{GitHub/TorchCraftExcelMod}}
        		{Java, Minecraft Forge}
        		{
        		Mod met de mogelijkheid om de "waarde" van je inventory te berekenen
        		\begin{itemize}
        			\item Meerdere instellingen
        			\item Exporteren naar csv-bestanden
        			\item Het laden van item data kan via configs
        		\end{itemize}}\\
\end{twenty}

\end{document}