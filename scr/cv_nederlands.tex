%%%%%%%%%%%%%%%%%%%%%%%%%%%%%%%%%%%%%%%%%
% Twenty Seconds Resume/CV
% LaTeX Template
% Version 1.0 (14/7/16)
%
% Original author:
% Carmine Spagnuolo (cspagnuolo@unisa.it) with major modifications by
% Vel (vel@LaTeXTemplates.com), Harsh (harsh.gadgil@gmail.com),
% ZabuzaW (zabuza.dev@gmail.com) and Tais993
%
%%%%%%%%%%%%%%%%%%%%%%%%%%%%%%%%%%%%%%%%%

%----------------------------------------------------------------------------------------
%	PACKAGES AND OTHER DOCUMENT CONFIGURATIONS
%----------------------------------------------------------------------------------------

\documentclass[letterpaper]{twentysecondcv} % a4paper for A4

%----------------------------------------------------------------------------------------
%	 PERSONAL INFORMATION
%----------------------------------------------------------------------------------------
% If you don't need one or more of the below, just remove the content leaving the command, e.g. \cvnumberphone{}

\cvname{Tijs Beek} % Your name
\cvjobtitle{Software Developer, 17} % Job
\cvavatar{avatar.jpg} % Avatar
% title/career

\cvaddress{Uden, Nederland} % Address
\cvlinkedin{/in/tijs-beek-ab038217b/} % LinkedIn site
\cvgithub{Tais993} % Github site
\cvgithubRepos{27} % GitHub amount of repositories
\cvgithubForks{2} % GitHub amount of forks
\cvgithubStars{3} % GitHub amount of stars
\cvsite{tais993.github.io/} % Personal website
\cvnumberphone{} % Phone number
\cvstackoverflow{} % StackOverflow user name
\cvstackoverflowId{} % StackOverflow user id
\cvstackoverflowRep{} % StackOverflow user reputation
\cvstackoverflowGold{} % StackOverflow amount of gold badges
\cvstackoverflowSilver{} % StackOverflow amount of silver badges
\cvstackoverflowBronze{} % StackOverflow amount of bronze badges
\cvmail{} % Email address


% Programming skill bars
\programming{
        {HTML $\textbullet$ CSS $\textbullet$ JS\slash TS $\textbullet$ PHP / 1.5},
        {C\# $\textbullet$ Kotlin / 3},
        {Java / 6}}

% Language overview bubbles
\newcommand\languages{
    ~
    \smartdiagram[bubble diagram]{
        \textbf{~~~Nederlands~~~}\\\textbf{oorspronkelijk},
        \textbf{~~~~~~~~Engels~~~~~~~~}\\\textbf{B2}
    }
}

% Technologies
\technologies{
        {materialteal!35/Java/Gson, Jackson, Jetbrains Annotations\vspace{1mm}\\
    JDA, Caffeine, JUnit, Logback, SLF4J, \vspace{1mm}\\
    Log4j},
        {materialteal!35/C\#/WPF, XUnit, Moq, NewtonsoftJson, \vspace{1mm}\\
    Jetbrains Annotations},
        {materialteal!35/Spring/Boot, Framework, Web, WebClient, \vspace{1mm}\\
    Reactor, JPA},
        {materialteal!35/Frontend/TailwindCSS, ReactJS},
        {materialteal!35/Databases/MongoDB, PostgreSQL},
        {materialteal!35/Networking/REST, JSON},
        {materialamber!20/IDEs/IntelliJ, Rider, PhpStorm, VSC, \vspace{1mm}\\
    Visual Studio},
        {materialamber!20/Versioning/GitHub, GitLab, GitKraken, BitBucket \vspace{1mm}\\
    Git},
        {materialamber!20/Build Management/Gradle, Maven, Resharper},
        {materialamber!20/Organization/GitHub, Space, Atlassian},
        {materialamber!20/Communication/Slack, Discord, Skype, MS Teams, \vspace{1mm}\\
    MS Outlook},
        {materialamber!20/Documentation/Markdown, Asciidoc, Javadoc},
        {materialamber!20/CI\/CD/Bamboo, GitHub actions},
        {materialamber!20/Issue Tracking/Jira, GitHub Issues, Space, Trello}}


%----------------------------------------------------------------------------------------

\begin{document}

    \makesidebarFirst % Print the sidebar

%----------------------------------------------------------------------------------------
%	 Profile
%----------------------------------------------------------------------------------------


    \section{Profiel}

    \begin{itemize}
        \item \textbf{Twee jaar} programmeer ervaring
        \item \textbf{Expert in Java}, kennis van C\#, Kotlin en Fullstack kennis
        \item \textbf{Enthousiast} in programmeren
    \end{itemize}

    \vspace{6mm}


%----------------------------------------------------------------------------------------
%	 EDUCATION
%----------------------------------------------------------------------------------------


    \section{Onderwijs}

    \begin{twenty} % Environment for a list with descriptions
        \twentyitem
        {2021 -}
        {Heden}
        {Software developer (bol) niveau 4}
        {\href{https://www.summacollege.nl/}{Summa College, Eindhoven}}
        {}
        {
            Ben van klas veranderd naar een differentatie klas, dit geeft me de mogelijk om sneller door de opleiding te gaan.
        }
        %\twentyitem{<dates>}{<title>}{<organization>}{<location>}{<description>}
    \end{twenty}

%----------------------------------------------------------------------------------------
%	 Hobbies
%----------------------------------------------------------------------------------------


    \section{Hobby's}

    \begin{twenty} % Environment for a list with descriptions
        \twentyitem
        {2012 -}
        {Heden}
        {Scouting Uotha}
        {\href{https://www.uotha.nl/}{Scouting Uotha}}
        {}
        { Ben er jaren geleden mee begonnen, en ik ga er nog steeds naartoe met een grote glimlach.}\\
        \twentyitem
        {Feb 2019 -}
        {Heden}
        {Programmeren}
        {\href{https://tais993.github.io/}{GitHub Pages}}
        {}
        {
            Bijna 2 jaar geleden begonnen met actief programmeren, pas in Augustus van 2021 begonnen met Software Development op school.
        }
    \end{twenty}

    \vspace{6mm}


%----------------------------------------------------------------------------------------
%	 Experience
%----------------------------------------------------------------------------------------


    \section{Ervaringen}
    \begin{twenty} % Environment for a list with descriptions
        \twentyitem
        {Feb 2022 -}
        {Heden}
        {Stage bij Vanderlande}
        {\href{https://www.vanderlande.com/}{Vanderlandestraat, Veghel}}
        {}
        {
            Ik werk aan een interne tool die te maken heeft met CI/CD, deze tool is geschreven in Java.
        }                   \\
        \twentyitem
        {Feb 2021 -}
        {Heden}
        {Bezorger bij de Kwalitaria}
        {\href{https://kwalitaria.nl/uden-zuid/uden-zuid/}{Kwalitaria, Uden-Zuid}}
        {}
        {Voedselleveringen} \\
        \twentyitem
        {Mar 2018 -}
        {Apr 2021}
        {Leerlingenraad Udenscollege}
        {\href{https://www.udenscollege.nl/vmbo/startpagina-vmbo/}{Udenscollege VMBO}}
        {}
        {
            Gedurende 3 jaar ben ik lid geweest van de leerlingenraad, waaronder 1+ jaar als voorzitter.
        }
        %\twentyitem{<dates>}{<title>}{<location>}{<description>}
    \end{twenty}

    \vspace{6mm}


%----------------------------------------------------------------------------------------
%	 Projects
%----------------------------------------------------------------------------------------


    \section{Projecten}


    \begin{twenty} % Environment for a list with descriptions
        \twentyitem
        {Maart 2022 -}
        {Heden}
        {Hayame}
        {\href{https://github.com/Tais993/Hayame}{GitHub/Hayame}}
        {Java 17, Gradle, MariaDB, Flyway, HikariCP, SQL, Prometheus}
        {
            Nog een basic Discord bot. Maar dit keer met lokalisatie! Het implementeren hiervan was makkelijker dan ik had verwacht.
            De bot heeft ook statestieken geimplementeerd door middel van Prometheus, en dit is mijn eerste SQL project (MariaDB + HikariCP)
        }                                                                                                   \\

        \twentyitem
        {Nov 2021 -}
        {Dec 2021}
        {HTML-Validator}
        {\href{https://github.com/Tais993/HTML-Validator/}{GitHub/HTML-Validator}}
        {Java 17, Gradle}
        {
            Dit project is niet afgerond, het is een groot project waar ik weinig van leerde.\\

        Het idee van deze library is om HTML te valideren, waarna het waarschuwingen en fouten terug geeft.\\
        Ik wou erna een website bij maken.}                                                                 \\

    \end{twenty}

    \newpage

    \makesidebarSecond % Print the sidebar

%----------------------------------------------------------------------------------------
%	 Projects
%----------------------------------------------------------------------------------------


    \section{Projecten}
    \begin{twenty} % Environment for a list with descriptions

        \twentyitem
        {Okt 2021 -}
        {Heden}
        {Java-OsuApiWrapper}
        {\href{https://github.com/Tais993/Java-OsuApiV1/}{GitHub/Java-OsuApiV1}}
        {Java 17, Gradle, Spring Boot Web, WebClient, Reactor Spring, Caffeine, IntelliJ Annotations, Jackson}
        {
            Dit project is still WIP

            Een wrapper voor de osu!api, dit is een REST API voor osu! de game                                  \\
        Rework van een van mijn oudere projecten.

        Dit project heeft me goed geholpen met het designen van libraries en het maken van goede documentatie.
            \begin{itemize}
                \item Zeer modificeerbaar
                \item Duidelijke documentatie
                \item Reactive
            \end{itemize}
        }\\

        \twentyitem
        {Aug 2021 -}
        {Heden}
        {TJ-Bot}
        {\href{https://github.com/Together-Java/TJ-Bot/}{(Klik voor meer info) GitHub/TJ-Bot}}
        {Java 17, Gradle, Jetbrains Annotations, JDA, JUnit, SLF4J, Logback, SQLite, Sonarcloud, Spotless}
        {Discord bot voor de zogenaamde "Together Java" Discord server, de bot wordt onderhouden door de community. \\
        Dit heeft mijn Git en samenwerkings vaardigheden versterkt in grote mate                                    \\
        Ik ben steeds een van de actiefste deelnemers van het project.
        }\\

        \twentyitem
        {Sep 2021 -}
        {Oct 2021}
        {VeelPlezier}
        {\href{https://github.com/Tais993/VeelPlezier}{GitHub/VeelPlezier}}
        {C\#, .NET framework 7.2, WPF, JetBrains Annotations, NewtonsoftJson, XUnit, Moq}
        {
            Dit was een project gekregen van school.

            Digitale kassa, eerste C\# project en eerste project met unit testing, \\

        Ik heb vooral geleerd hoe C\# werkt, unit testing zat al in mijn hoofd, heb het toegepast in dit project.
            \begin{itemize}
                \item Responsieve GUI
                \item Lokalisatie
                \item Inclusief rekenmachine
                \item Inclusief valuta omzetter
            \end{itemize}
        }\\

        \twentyitem
        {June 2020 -}
        {Sep 2021}
        {TorchCraftExcelMod}
        {\href{https://github.com/Tais993/TorchCraftExcelMod}{GitHub/TorchCraftExcelMod}}
        {Java, Minecraft Forge}
        {
            Minecraft mod voor prison spelers.                                                                          \\
            Hoofd functionaliteit is het berekenen hoeveel de items in je inventory waard zijn in de servers munteenheid
            \begin{itemize}
                \item Meerdere instellingen
                \item Exporteren naar csv-bestanden
                \item Het laden van item data kan via configs
            \end{itemize}
        }\\
    \end{twenty}
\end{document}
\begin{document}

\end{document}