%%%%%%%%%%%%%%%%%%%%%%%%%%%%%%%%%%%%%%%%%
% Twenty Seconds Resume/CV
% LaTeX Template
% Version 1.0 (14/7/16)
%
% Original author:
% Carmine Spagnuolo (cspagnuolo@unisa.it) with major modifications by
% Vel (vel@LaTeXTemplates.com), Harsh (harsh.gadgil@gmail.com),
% ZabuzaW (zabuza.dev@gmail.com) and Tais993
%
%%%%%%%%%%%%%%%%%%%%%%%%%%%%%%%%%%%%%%%%%

%----------------------------------------------------------------------------------------
%	PACKAGES AND OTHER DOCUMENT CONFIGURATIONS
%----------------------------------------------------------------------------------------

\documentclass[letterpaper]{twentysecondcv} % a4paper for A4

%----------------------------------------------------------------------------------------
%	 PERSONAL INFORMATION
%----------------------------------------------------------------------------------------
% If you don't need one or more of the below, just remove the content leaving the command, e.g. \cvnumberphone{}

\cvname{Tijs Beek} % Your name
\cvjobtitle{Software Developer, 17} % Job
\cvavatar{avatar.jpg} % Avatar
% title/career

\cvaddress{Uden, Netherlands} % Address
\cvlinkedin{/in/tijs-beek-ab038217b/} % LinkedIn site
\cvgithub{Tais993} % Github site
\cvgithubRepos{27} % GitHub amount of repositories
\cvgithubForks{2} % GitHub amount of forks
\cvgithubStars{3} % GitHub amount of stars
\cvsite{tais993.github.io/} % Personal website
\cvnumberphone{} % Phone number
\cvstackoverflow{} % StackOverflow user name
\cvstackoverflowId{} % StackOverflow user id
\cvstackoverflowRep{} % StackOverflow user reputation
\cvstackoverflowGold{} % StackOverflow amount of gold badges
\cvstackoverflowSilver{} % StackOverflow amount of silver badges
\cvstackoverflowBronze{} % StackOverflow amount of bronze badges
\cvmail{} % Email address


% Programming skill bars
\programming{
        {HTML $\textbullet$ CSS $\textbullet$ JS\slash TS $\textbullet$ PHP / 1.5},
        {C\# $\textbullet$ Kotlin / 3},
        {Java / 6}}

% Language overview bubbles
\newcommand\languages{
    ~
    \smartdiagram[bubble diagram]{
        \textbf{~~~Dutch~~~}\\\textbf{native},
        \textbf{~~~~~~~~English~~~~~~~~}\\\textbf{B2}
    }
}

% Technologies
\technologies{
        {materialteal!35/Java/Gson, Jackson, Jetbrains Annotations\vspace{1mm}\\
    JDA, Caffeine, JUnit, Logback, SLF4J, \vspace{1mm}\\
    Log4j},
        {materialteal!35/C\#/WPF, XUnit, Moq, NewtonsoftJson, \vspace{1mm}\\
    Jetbrains Annotations},
        {materialteal!35/Spring/Boot, Framework, Web, WebClient, \vspace{1mm}\\
    Reactor, JPA},
        {materialteal!35/Frontend/TailwindCSS, ReactJS},
        {materialteal!35/Databases/MongoDB, PostgreSQL},
        {materialteal!35/Networking/REST, JSON},
        {materialamber!20/IDEs/IntelliJ, Rider, PhpStorm, VSC, \vspace{1mm}\\
    Visual Studio},
        {materialamber!20/Versioning/GitHub, GitLab, GitKraken, BitBucket \vspace{1mm}\\
    Git},
        {materialamber!20/Build Management/Gradle, Maven, Resharper},
        {materialamber!20/Organization/GitHub, Space, Atlassian},
        {materialamber!20/Communication/Slack, Discord, Skype, MS Teams, \vspace{1mm}\\
    MS Outlook},
        {materialamber!20/Documentation/Markdown, Asciidoc, Javadoc},
        {materialamber!20/CI\/CD/Bamboo, GitHub actions}}


%----------------------------------------------------------------------------------------

\begin{document}

    \makesidebarFirst % Print the sidebar

%----------------------------------------------------------------------------------------
%	 Profile
%----------------------------------------------------------------------------------------


    \section{Profile}

    \begin{itemize}
        \item \textbf{Two years} of programming experience
        \item \textbf{Expert in Java}, knowledged in C\#, Kotlin and Fullstack knowledge
        \item \textbf{Enthusiastic} in programming
    \end{itemize}

    \vspace{6mm}


%----------------------------------------------------------------------------------------
%	 EDUCATION
%----------------------------------------------------------------------------------------


    \section{Education}

    \begin{twenty} % Environment for a list with descriptions
        \twentyitem
        {2021 -}
        {Present~~~~}
        {Software developer (bol) level 4}
        {\href{https://www.summacollege.nl/}{Summa College, Eindhoven}}
        {}
        {
            Switched class into an differentiation class, this allows me to \\ advance at a faster pace through the study.
        }
        %\twentyitem{<dates>}{<title>}{<organization>}{<location>}{<description>}
    \end{twenty}

%----------------------------------------------------------------------------------------
%	 Hobbies
%----------------------------------------------------------------------------------------


    \section{Hobbies}

    \begin{twenty} % Environment for a list with descriptions
        \twentyitem
        {2012 -}
        {Present~~~~}
        {Scouting Uotha}
        {\href{https://www.uotha.nl/}{Scouting Uotha}}
        {}
        { Started years ago, and still going with a big smile.}\\
        \twentyitem
        {Feb 2019 -}
        {Present~~~~}
        {Programming}
        {\href{https://tais993.github.io/}{GitHub Pages}}
        {}
        {Started almost 2 years ago by myself, and only since August 2021 started with Software Development at school.}
    \end{twenty}

    \vspace{6mm}


%----------------------------------------------------------------------------------------
%	 Experience
%----------------------------------------------------------------------------------------


    \section{Experience}
    \begin{twenty} % Environment for a list with descriptions
        \twentyitem
        {Feb 2022 -}
        {Present}
        {Internship at Vanderlande}
        {\href{https://www.vanderlande.com/}{Vanderlandestraat, Veghel}}
        {}
        {
            I'm working on an internal tool surrounding CI/CD
        }\\
        \twentyitem
        {Feb 2021 -}
        {Present}
        {Deliverer at Kwalitaria}
        {\href{https://kwalitaria.nl/uden-zuid/uden-zuid/}{Kwalitaria, Uden-Zuid}}
        {}
        {Food delivery}\\
        \twentyitem
        {Mar 2018 -}
        {Apr 2021}
        {Student council Udenscollege}
        {\href{https://www.udenscollege.nl/vmbo/startpagina-vmbo/}{Udenscollege VMBO}}
        {}
        {
            For over 3 years I've been a member of the student-council, including 1+ year as chairman.
        }
        %\twentyitem{<dates>}{<title>}{<location>}{<description>}
    \end{twenty}

    \vspace{6mm}


%----------------------------------------------------------------------------------------
%	 Projects
%----------------------------------------------------------------------------------------


    \section{Projects}
    \begin{twenty} % Environment for a list with descriptions
        \twentyitem
        {Nov 2021 -}
        {Dec 2021}
        {HTML-Validator}
        {\href{https://github.com/Tais993/HTML-Validator/}{GitHub/HTML-Validator}}
        {Java 17, Gradle}
        {
            This project is unfinished, it's a big project which I don't learn a lot from.\\

        The idea of this library is to validate HTML, response with possible warnings and errors.\\
        It was my intention to afterwards create a website for it.}\\

        \twentyitem
        {Oct 2021 -}
        {Present}
        {Java-OsuApiWrapper}
        {\href{https://github.com/Tais993/Java-OsuApiV1/}{GitHub/Java-OsuApiV1}}
        {Java 17, Gradle, Spring Boot Web, WebClient, Reactor Spring, Caffeine, IntelliJ Annotations, Jackson}
        {
            Note, this project is still WIP

        Wrapper for the osu-api, which is a REST API for osu! the game.
        Revamp of one of my older projects.\\

        I mainly learned how to design libraries and create good documentation.
            \begin{itemize}
                \item Highly customizable
                \item Clear documentation
                \item Reactive
            \end{itemize}}
    \end{twenty}

    \newpage

    \makesidebarSecond % Print the sidebar

%----------------------------------------------------------------------------------------
%	 Projects
%----------------------------------------------------------------------------------------


    \section{Projects}
    \begin{twenty} % Environment for a list with descriptions
        \twentyitem
        {Aug 2021 -}
        {Present}
        {TJ-Bot}
        {\href{https://github.com/Together-Java/TJ-Bot/}{(Click for more info) GitHub/TJ-Bot}}
        {Java 17, Gradle, Jetbrains Annotations, JDA, JUnit, SLF4J, Logback, SQLite, Sonarcloud, Spotless}
        {
            Discord bot for the so called "Together Java" Discord server, maintained by the community.\\
        This project enhanced my teamwork and Git skills to a great extent.\\
        Still one of the most active members of the project.
        }\\

        \twentyitem
        {Sep 2021 -}
        {Oct 2021}
        {VeelPlezier}
        {\href{https://github.com/Tais993/VeelPlezier}{GitHub/VeelPlezier}}
        {C\#, .NET framework 7.2, WPF, JetBrains Annotations, NewtonsoftJson, XUnit, Moq}
        {
            Project given to me by school.

            Digital checkout, first C\# project and first project with unit testing.\\
        I mainly taught how C\# works, unit testing was already in my head, only applied it here.

            \begin{itemize}
                \item Responsive GUI
                \item Localisation
                \item Includes calculator
                \item Includes currency converter
            \end{itemize}
        }\\

        \twentyitem
        {June 2020 -}
        {Sep 2021}
        {TorchCraftExcelMod}
        {\href{https://github.com/Tais993/TorchCraftExcelMod}{GitHub/TorchCraftExcelMod}}
        {Java, Minecraft Forge}
        {
            Minecraft Mod for prison players.
            The main functionality of the mod is to see how much their inventory items are worth.
            \begin{itemize}
                \item Settings
                \item Exporting to csv files
                \item Loading item data from external config file
            \end{itemize}
        }\\
    \end{twenty}

\end{document}